% v006 
\newpage


	
\section{خوشه‌بندی زبان از روی ویژگی‌های صوتی}

\subsection{تعریف مسئله و هدف}	هدف ، خوشه‌بندی نمونه‌های صوتی بر اساس زبان گفتار است؛ به این معنا که با استفاده از بردار ویژگی استخراج‌شده از هر فایل صوتی، بدون استفاده از برچسب‌ها (یادگیری بدون‌نظارت)، نمونه‌ها به چند خوشه تقسیم شوند. سپس برای تحلیل ، کیفیت خوشه‌بندی هم با معیارهای بدون‌نظارت و هم با معیارهای وابسته به برچسب واقعی (صرفاً برای ارزیابی) بررسی می‌شود.

\subsection{داده، ویژگی‌ها و آمار توصیفی}
\subsubsection{ساختار داده و برچسب‌ها}	بر اساس خروجی‌های ارسال‌شده در فایل \lr{outputs.zip}:
\begin{itemize}
\item تعداد کل نمونه‌ها $720$ است.
\item تعداد زبان‌ها $4$ است و توزیع داده در هر زبان یکنواخت است: برای هر زبان $180$ نمونه.
\item داده به دو حالت اصلی تقسیم شده است:
\begin{itemize}
	\item حالت \lr{no\_augmentation}: $\;X_{\text{train}}$ با ابعاد $\,(576,86)\,$ و $\;X_{\text{test}}$ با ابعاد $\,(144,86)\,$.
	\item حالت \lr{augmented}: $\;X_{\text{train}}$ با ابعاد $\,(3456,86)\,$ و $\;X_{\text{test}}$ با ابعاد $\,(144,86)\,$.
\end{itemize}
\end{itemize}

\subsubsection{ویژگی‌ها}	برای هر نمونه یک بردار ویژگی با تعداد $86$ ویژگی ساخته شده است. شمارش نوع ویژگی‌ها (از روی نام ویژگی‌ها) به صورت زیر است:
\begin{itemize}
\item ویژگی‌های \lr{MFCC} و مشتقات آن: $78$ ویژگی
\item ویژگی‌های طیفی \lr{spectral}: $4$ ویژگی
\item ویژگی‌های \lr{zero-crossing}: $2$ ویژگی
\item ویژگی‌های \lr{RMS}: $2$ ویژگی
\end{itemize}

\subsubsection{نرمال‌سازی داده (استانداردسازی)}	داده‌ی خامِ ویژگی‌ها از قبل استانداردسازی شده است (میانگین نزدیک به صفر و انحراف معیار نزدیک به یک). برای مثال در حالت \lr{no\_augmentation} روی داده‌ی آموزش:
\begin{itemize}
\item بازه‌ی میانگین ویژگی‌ها تقریباً بین $\,-5.80\times 10^{-15}$ تا $3.55\times 10^{-15}$.
\item بازه‌ی انحراف معیار ویژگی‌ها تقریباً بین $1.00$ تا $1.00$ (با خطای عددی بسیار کوچک).
\end{itemize}	این نکته مهم است، چون بسیاری از روش‌های فاصله‌محور (مثل \lr{K-Means} و روش‌های چگالی‌محور) نسبت به مقیاس ویژگی‌ها حساس هستند.

\subsubsection{مدت زمان فایل‌های صوتی}	ستون \lr{duration} نشان می‌دهد:
\begin{itemize}
\item کمینه‌ی مدت: $39.09$ ثانیه
\item میانگین مدت: $63.20$ ثانیه
\item بیشینه‌ی مدت: $1537.77$ ثانیه
\end{itemize}	(این ناهمگنی می‌تواند در کیفیت استخراج ویژگی و میزان نویز اثرگذار باشد.)

\subsection{چارچوب ارزیابی و معیارها}
\subsubsection{معیار \lr{Silhouette} (بدون‌نظارت)}	برای ارزیابی کیفیت خوشه‌بندی بدون استفاده از برچسب‌ها، از \lr{silhouette score} استفاده شده است. برای هر نمونه $i$:
\[
s(i)=\frac{b(i)-a(i)}{\max\{a(i),b(i)\}},
\]	که در آن $a(i)$ میانگین فاصله‌ی نمونه‌ی $i$ تا اعضای خوشه‌ی خودش و $b(i)$ کمترین میانگین فاصله‌ی $i$ تا نزدیک‌ترین خوشه‌ی دیگر است. مقدار نهایی، میانگین $s(i)$ روی همه‌ی نمونه‌هاست. تعریف رسمی در مستندات \lr{scikit-learn} آمده است.\footnote{\begin{latin}
{scikit-learn silhouette\_score documentation}:\\ \url{https://scikit-learn.org/stable/modules/generated/sklearn.metrics.silhouette_score.html}
\end{latin}
}

\subsubsection{معیار \lr{Purity} (ارزیابی با برچسب واقعی)}	هرچند خوشه‌بندی بدون‌نظارت است، برای تحلیل هم‌خوانی خوشه‌ها با زبان واقعی از \lr{purity} استفاده می‌کنیم:
\[
\text{Purity}=\frac{1}{n}\sum_{j=1}^{k}\max_{\ell}\left|C_j\cap L_\ell\right|,
\]	که $C_j$ مجموعه‌ی نقاط خوشه‌ی $j$ و $L_\ell$ مجموعه‌ی نقاط با برچسب واقعی $\ell$ است. (در روش‌های دارای نویز، \lr{purity} معمولاً روی نقاط غیرنویز گزارش می‌شود.)

\subsubsection{نسبت نویز در روش‌های چگالی‌محور}	در \lr{DBSCAN} و \lr{OPTICS} نقاطی که عضو هیچ خوشه‌ای نشوند با برچسب $-1$ (نویز) مشخص می‌شوند. بنابراین نسبت نویز:
\[
\text{NoiseRatio}=\frac{\#\{i:\; \text{label}(i)=-1\}}{n}.
\]	در \lr{OPTICS} با استخراج \lr{Xi} نیز نقاط خارج از خوشه‌ها با $-1$ برچسب می‌خورند.\footnote{
\begin{latin}
{scikit-learn cluster\_optics\_xi documentation}:\\ \url{https://scikit-learn.org/stable/modules/generated/sklearn.cluster.cluster_optics_xi.html}
\end{latin}
}

\subsection{پیش‌پردازش و ساخت دو نسخه‌ی داده ( \lr{Step$1$} )}	در \lr{Step$1$} برای هر حالت داده ( \lr{no\_augmentation} و \lr{augmented} ) دو نسخه ساخته شده است:
\begin{itemize}
\item نسخه‌ی \lr{RAW}: همان فضای اصلی با $86$ ویژگی استانداردشده.
\item نسخه‌ی \lr{OPTIMIZED}: داده‌ی تبدیل‌یافته با کاهش‌بُعد به کمک \lr{PCA}.
\end{itemize}

\subsubsection{بررسی هم‌بستگی و حذف ویژگی‌های با هم‌بستگی بالا}	در این نسخه از خروجی‌ها، آستانه‌ی حذف هم‌بستگی $\;0.99\;$ بوده و در هر دو حالت:
\begin{itemize}
\item تعداد ویژگی‌های حذف‌شده $0$ و تعداد ویژگی‌های نگه‌داشته‌شده $86$ است.
\end{itemize}	نکته‌ی مهم: با وجود این، در محاسبه‌ی هم‌بستگی روی \lr{X\_train} مشاهده می‌شود که در حالت \lr{no\_augmentation} تعداد $4$ جفت ویژگی با $\left|r\right|>0.95$ وجود دارد (ولی هیچ جفتی به $0.99$ نمی‌رسد)، لذا حذف انجام نشده است.

\subsubsection{کاهش‌بُعد با \lr{PCA}}	در \lr{Step$1$}، \lr{PCA} با هدف حفظ حدود $90\%$ واریانس اعمال شده است:
\begin{itemize}
\item در حالت \lr{no\_augmentation}: تعداد مؤلفه‌ها $30$ و مجموع نسبت واریانس توضیح‌داده‌شده $0.9052$.
\item در حالت \lr{augmented}: تعداد مؤلفه‌ها $32$ و مجموع نسبت واریانس توضیح‌داده‌شده $0.9007$.
\end{itemize}	در این خروجی‌ها گزینه‌ی \lr{whiten} برابر \lr{False} بوده است؛ بنابراین \lr{PCA} صرفاً یک نگاشت خطی برای بیشینه‌سازی واریانس روی مؤلفه‌هاست و مقیاس‌دهی سفیدسازی انجام نشده است.
	% --- جای‌گذاری شکل‌های Step1 ---
	

\subsection{خوشه‌بندی با \lr{K-Means} ( \lr{Step$2$} )}
\subsubsection{مدل و تابع هدف}	الگوریتم \lr{K-Means} داده‌ها را به $k$ خوشه تقسیم می‌کند و مراکز $\mu_j$ را طوری می‌یابد که مجموع مربعات فاصله‌ها کمینه شود:
\[
J=\sum_{i=1}^{n}\left\lVert x_i-\mu_{c(i)}\right\rVert_2^2.
\]	در \lr{scikit-learn} این مقدار با نام \lr{inertia} گزارش می‌شود.\footnote{
\begin{latin}
{scikit-learn KMeans documentation}: \url{https://scikit-learn.org/stable/modules/generated/sklearn.cluster.KMeans.html}
\end{latin}
}

\subsubsection{روش انتخاب $k$}	در خروجی‌های \lr{Step$2$}، مقدار $k$ در بازه‌ی $\{2,\dots,9\}$ پیمایش شده و برای انتخاب بهترین $k$ از بیشینه‌سازی \lr{silhouette} استفاده شده است. همچنین نمودار \lr{elbow} (بر اساس \lr{inertia}) برای مشاهده‌ی تغییرات تابع هدف ذخیره شده است.

\subsubsection{نتایج عددی (بهترین $k$)}	جدول زیر از فایل‌های \lr{summary.json} در \lr{Step$2$} استخراج شده است:

\begin{table}[h]
\centering
\caption{نتایج بهترین تنظیم \lr{K-Means} بر اساس بیشینه‌سازی \lr{silhouette} ( \lr{Step$2$} )}
\begin{tabular}{lllll}
	\toprule			حالت داده & نسخه & $k^\star$ & $\text{Silhouette}$ & $\text{Purity}$ \\
		\midrule
	\lr{no\_augmentation} & \lr{RAW} & $4$ & $0.1938$ & $0.6181$ \\
	\lr{no\_augmentation} & \lr{OPT} & $4$ & $0.2148$ & $0.6181$ \\
	\lr{augmented} & \lr{RAW} & $5$ & $0.1460$ & $0.5483$ \\
	\lr{augmented} & \lr{OPT} & $4$ & $0.1583$ & $0.4627$ \\
	\bottomrule
\end{tabular}
\end{table}

\paragraph{تفسیر.}
\begin{itemize}
\item در حالت \lr{no\_augmentation}، هر دو نسخه‌ی \lr{RAW} و \lr{OPT} بهترین $k$ را برابر $4$ برمی‌گردانند (هم‌خوان با تعداد زبان‌ها) و \lr{silhouette} در نسخه‌ی \lr{OPT} بهتر شده است.
\item در حالت \lr{augmented}، نسخه‌ی \lr{RAW} بهترین $k$ را $5$ می‌دهد، اما نسخه‌ی \lr{OPT} بهترین $k$ را $4$ بازمی‌گرداند و \lr{silhouette} نیز بهتر می‌شود؛ با این حال \lr{purity} در نسخه‌ی \lr{OPT} کاهش یافته است. این می‌تواند نشان دهد که کاهش‌بُعد با \lr{PCA} هندسه‌ی خوشه‌ها را از نظر فاصله‌ای بهتر کرده، اما هم‌راستایی خوشه‌ها با زبان واقعی را (به علت حذف مؤلفه‌های کم‌واریانس ولی شاید تمایزبخش) کاهش داده است.
\end{itemize}
% --- جای‌گذاری شکل‌های Step2 (مسیرها را مطابق پوشه‌ی خودتان اصلاح کنید) ---
 \begin{figure}[h]
 \centering
  \includegraphics[width=0.95\linewidth]{../img/outputs/step2_kmeans_compare/no_augmentation/compare/figures/silhouette_side_by_side.png}
 \caption{ مقایسه‌ی \lr{silhouette} بر حسب $k$ برای \lr{K-Means} در \lr{RAW} و \lr{OPT} (حالت \lr{no\_augmentation}).}
 \end{figure}

 \begin{figure}[h]
 \centering
  \includegraphics[width=0.95\linewidth]{../img/outputs/step2_kmeans_compare/augmented/compare/figures/elbow_side_by_side.png}
 \caption{ نمودار \lr{elbow} (بر اساس \lr{inertia}) برای \lr{K-Means} در \lr{RAW} و \lr{OPT} (حالت \lr{augmented}).}
 \end{figure}

 \begin{figure}[h]
 \centering
  \includegraphics[width=0.95\linewidth]{../img/outputs/step2_kmeans_compare/no_augmentation/compare/figures/raw_truth_vs_kmeans.png}
 \caption{ نمای \lr{PCA2}: برچسب واقعی زبان در کنار برچسب خوشه‌ی \lr{K-Means} (نسخه‌ی \lr{RAW}).}
 \end{figure}

\subsection{خوشه‌بندی چگالی‌محور با \lr{DBSCAN} ( \lr{Step$3$} )}
\subsubsection{ایده‌ی اصلی و تعریف نقاط هسته/مرزی/نویز}	الگوریتم \lr{DBSCAN} خوشه‌ها را بر اساس چگالی تعریف می‌کند. برای یک نقطه $x$، همسایگی $\varepsilon$ به صورت
\[
\mathcal{N}_\varepsilon(x)=\{y:\; d(x,y)\le \varepsilon\}
\]	تعریف می‌شود. اگر $\left|\mathcal{N}_\varepsilon(x)\right|\ge \text{min\_samples}$، نقطه \textbf{هسته} است و خوشه با اتصال نقاط هسته توسعه می‌یابد. نقاطی که به هیچ خوشه‌ای نپیوندند \textbf{نویز} بوده و با برچسب $-1$ مشخص می‌شوند.\footnote{
\begin{latin}
\lr{scikit-learn DBSCAN documentation}:\\ \url{https://scikit-learn.org/stable/modules/generated/sklearn.cluster.DBSCAN.html}
\end{latin}
}

\subsubsection{روش جست‌وجوی پارامترها}	در \lr{Step$3$} :
\begin{itemize}
\item روی \lr{min\_samples} یک شبکه از مقادیر گسسته بررسی شده است.
\item برای هر \lr{min\_samples}، با استفاده از منحنی \lr{k-distance} و چند \lr{percentile}، مقادیر کاندید برای \lr{eps} تولید شده است.
\item \lr{silhouette} روی نقاط غیرنویز محاسبه شده و در انتخاب تنظیم بهتر، علاوه بر نزدیک بودن تعداد خوشه‌ها به $4$، محدودیت روی نسبت نویز نیز در نظر گرفته شده است.
\end{itemize}

\subsubsection{نتایج عددی بهترین تنظیم‌ها}	(مقادیر از فایل‌های \lr{best\_params.json} در \lr{Step$3$} استخراج شده‌اند.)

\begin{table}[h]
\centering
\caption{بهترین تنظیم‌های \lr{DBSCAN} و کیفیت ( \lr{Step$3$} )}
	\resizebox{\textwidth}{!}{%
	\begin{tabular}{lllllll}
		\toprule			حالت & نسخه & $\varepsilon$ & $\text{min\_samples}$ & $\#\text{clusters}$ & $\text{NoiseRatio}$ & $\text{Silhouette}_{\neg \text{noise}}$ \\
		\midrule
		\lr{no\_augmentation} & \lr{RAW} & $10.0007$ & $5$  & $4$ & $0.0174$ & $0.1755$ \\
		\lr{no\_augmentation} & \lr{OPT} & $6.9179$  & $5$  & $4$ & $0.1406$ & $0.2138$ \\
		\lr{augmented} & \lr{RAW} & $6.5405$ & $15$ & $4$ & $0.2338$ & $0.1617$ \\
		\lr{augmented} & \lr{OPT} & $5.6557$ & $15$ & $4$ & $0.2338$ & $0.1571$ \\
		\bottomrule
	\end{tabular}
}
\end{table}

\begin{table}[h]
\centering
\caption{\lr{Purity} روی نقاط غیرنویز و دامنه‌ی اندازه‌ی خوشه‌ها در \lr{DBSCAN} ( \lr{Step$3$} )}
\begin{tabular}{lllll}
	\toprule			حالت & نسخه & $\text{Purity}_{\neg \text{noise}}$ & $\min |C_j|$ & $\max |C_j|$ \\
		\midrule
	\lr{no\_augmentation} & \lr{RAW} & $0.2968$ & $6$  & $541$ \\
	\lr{no\_augmentation} & \lr{OPT} & $0.4263$ & $7$  & $420$ \\
	\lr{augmented} & \lr{RAW} & $0.4060$ & $30$ & $2341$ \\
	\lr{augmented} & \lr{OPT} & $0.4052$ & $30$ & $2340$ \\
	\bottomrule
\end{tabular}
\end{table}

\paragraph{تفسیر.}	در همه‌ی حالت‌ها \lr{DBSCAN} توانسته تعداد خوشه‌ها را $4$ تولید کند؛ اما \lr{purity} نسبتاً پایین است. همچنین در حالت \lr{augmented} یک خوشه‌ی بسیار بزرگ (بیش از $2000$ نقطه) دیده می‌شود که می‌تواند نشانه‌ی ادغام چند زبان/گوینده در یک ناحیه‌ی چگال یا انتخاب \lr{eps} نسبتاً بزرگ باشد.
	% --- جای‌گذاری شکل‌های Step3 ---
 \begin{figure}[h]
 \centering
  \includegraphics[width=0.95\linewidth]{../img/outputs/step3_dbscan_compare/no_augmentation/compare/figures/dbscan_side_by_side.png}
 \caption{ مقایسه‌ی نمای \lr{PCA2} خوشه‌های بهترین \lr{DBSCAN} برای \lr{RAW} و \lr{OPT}.}
 \end{figure}

 \begin{figure}[h]
 \centering
  \includegraphics[width=0.95\linewidth]{../img/outputs/step3_dbscan_compare/augmented/optimized/figures/kdist_ms_15.png}
 \caption{ منحنی \lr{k-distance} برای تخمین \lr{eps} (نمونه، حالت \lr{augmented-OPT}).}
 \end{figure}

 \begin{figure}[h]
 \centering
  \includegraphics[width=0.95\linewidth]{../img/outputs/step3_dbscan_compare/no_augmentation/compare/figures/raw_truth_vs_dbscan.png}
 \caption{ نمای \lr{PCA2}: برچسب واقعی زبان در کنار برچسب خوشه‌ی \lr{DBSCAN} (نسخه‌ی \lr{RAW}).}
 \end{figure}

\subsection{خوشه‌بندی با \lr{OPTICS} ( \lr{Step$4$} )}
\subsubsection{ایده‌ی اصلی و نقش \lr{reachability}}	الگوریتم \lr{OPTICS} نیز چگالی‌محور است، اما به جای تعیین یک \lr{eps} ثابت، ساختار چگالی را در مقیاس‌های مختلف ثبت می‌کند و خروجی کلیدی آن \lr{reachability distance} است. با رسم \lr{reachability plot} معمولاً «دره‌ها» متناظر با خوشه‌ها تفسیر می‌شوند. نمونه و توضیح رسمی در مثال‌های \lr{scikit-learn} موجود است.\footnote{
\begin{latin}
{scikit-learn OPTICS example (reachability plot)}:\\ \url{https://scikit-learn.org/stable/auto_examples/cluster/plot_optics.html}
\end{latin}
}

\subsubsection{استخراج خوشه‌ها با روش \lr{Xi}}	در این پروژه خوشه‌ها با روش \lr{Xi} استخراج شده‌اند که پارامتر \lr{xi} حداقل تندیِ لازم روی \lr{reachability plot} برای مرز خوشه را تعیین می‌کند و نقاط خارج از خوشه‌ها با $-1$ برچسب می‌خورند.\footnote{
\begin{latin}
{scikit-learn cluster\_optics\_xi documentation}:\\ \url{https://scikit-learn.org/stable/modules/generated/sklearn.cluster.cluster_optics_xi.html}
\end{latin}
}

\subsubsection{نتایج عددی بهترین تنظیم‌ها}	(مقادیر از فایل‌های \lr{best\_params.json} در \lr{Step$4$} استخراج شده‌اند.)

\begin{table}[h]
\centering
\caption{بهترین تنظیم‌های \lr{OPTICS} و کیفیت ( \lr{Step$4$} )}
\resizebox{\textwidth}{!}{
\begin{tabular}{lllllll}
	\toprule			حالت & نسخه & $\text{min\_samples}$ & $\xi$ & $\#\text{clusters}$ & $\text{NoiseRatio}$ & $\text{Silhouette}_{\neg \text{noise}}$ \\
		\midrule
	\lr{no\_augmentation} & \lr{RAW} & $15$ & $0.08$ & $4$ & $0.8142$ & $0.4298$ \\
	\lr{no\_augmentation} & \lr{OPT} & $20$ & $0.05$ & $4$ & $0.7569$ & $0.5578$ \\
	\lr{augmented} & \lr{RAW} & $5$  & $0.03$ & $2$ & $0.9248$ & $0.4869$ \\
	\lr{augmented} & \lr{OPT} & $5$  & $0.03$ & $2$ & $0.8843$ & $0.4704$ \\
	\bottomrule
\end{tabular}}
\end{table}

\begin{table}[h]
\centering
\caption{\lr{Purity} روی نقاط غیرنویز در \lr{OPTICS} ( \lr{Step$4$} )}
\resizebox{\textwidth}{!}{
\begin{tabular}{llll}
	\toprule			حالت & نسخه & $\text{Purity}_{\neg \text{noise}}$ & توضیح کلیدی \\
		\midrule
	\lr{no\_augmentation} & \lr{RAW} & $1.0000$ & نویز بسیار زیاد، خوشه‌های غیرنویز بسیار خالص \\
	\lr{no\_augmentation} & \lr{OPT} & $0.9929$ & نویز زیاد، بهبود \lr{silhouette} در \lr{OPT} \\
	\lr{augmented} & \lr{RAW} & $1.0000$ & فقط $2$ خوشه + نویز بسیار زیاد \\
	\lr{augmented} & \lr{OPT} & $0.4125$ & هم تعداد خوشه کم، هم هم‌خوانی پایین‌تر \\
	\bottomrule
\end{tabular}}
\end{table}

\paragraph{تفسیر.}
\begin{itemize}
\item در حالت \lr{no\_augmentation}، \lr{OPTICS} تعداد خوشه را $4$ بازمی‌گرداند و روی نقاط غیرنویز، \lr{purity} تقریباً کامل است؛ اما نسبت نویز بسیار بالا است (بیش از $0.75$). این یعنی الگوریتم فقط هسته‌های بسیار متراکم را به عنوان خوشه تشخیص داده و بخش بزرگی از داده را نامطمئن (نویز) فرض کرده است.
\item در حالت \lr{augmented}، بهترین تنظیم‌ها فقط $2$ خوشه را استخراج کرده‌اند و نسبت نویز حتی بالاتر است. این می‌تواند نشان دهد که افزایش تنوع داده، ساختار چگالی را طوری تغییر داده که روش استخراج \lr{Xi} «چهار دره‌ی پایدار» روی \lr{reachability plot} پیدا نمی‌کند و خوشه‌بندی محافظه‌کارانه می‌شود.
\end{itemize}
% --- جای‌گذاری شکل‌های Step4 ---
 \begin{figure}[h]
 \centering
  \includegraphics[width=0.95\linewidth]{../img/outputs/step4_optics_compare/no_augmentation/compare/figures/reachability_side_by_side.png}
 \caption{ مقایسه‌ی \lr{reachability plot} برای \lr{RAW} و \lr{OPT} (حالت \lr{no\_augmentation}).}
 \end{figure}

 \begin{figure}[h]
 \centering
  \includegraphics[width=0.95\linewidth]{../img/outputs/step4_optics_compare/augmented/compare/figures/reachability_side_by_side.png}
 \caption{ مقایسه‌ی \lr{reachability plot} برای \lr{RAW} و \lr{OPT} (حالت \lr{augmented}).}
 \end{figure}

 \begin{figure}[h]
 \centering
  \includegraphics[width=0.95\linewidth]{../img/outputs/step4_optics_compare/no_augmentation/compare/figures/opt_truth_vs_optics.png}
 \caption{ نمای \lr{PCA2}: برچسب واقعی زبان در کنار برچسب خوشه‌ی \lr{OPTICS} (نسخه‌ی \lr{OPT}).}
 \end{figure}

\subsection{جمع‌بندی مقایسه‌ای تا پایان \lr{Step$4$}}	برای نتیجه‌گیری، یک دید خلاصه از خروجی هر الگوریتم ارائه می‌کنیم. در این جدول، برای \lr{K-Means} نسبت نویز تعریف نمی‌شود (علامت \lr{---})، اما برای \lr{DBSCAN} و \lr{OPTICS} گزارش شده است.

\begin{table}[H]
	\centering
	\caption{مقایسه‌ی خلاصه‌ی بهترین خروجی‌ها در سه الگوریتم (تا \lr{Step}$4$)}
	\resizebox{\textwidth}{!}{%
		\begin{tabular}{llllllll}
			\toprule			
			الگوریتم & حالت & نسخه & $\#\text{clusters}$ & معیار انتخاب & $\text{Silhouette}$ & $\text{Purity}$ & $\text{NoiseRatio}$ \\
			\midrule
			\lr{K-Means} & \lr{no\_augmentation} & \lr{RAW} & $4$ & $\max$ \lr{silhouette} & $0.1938$ & $0.6181$ & \lr{---} \\
			\lr{K-Means} & \lr{no\_augmentation} & \lr{OPT} & $4$ & $\max$ \lr{silhouette} & $0.2148$ & $0.6181$ & \lr{---} \\
			\lr{K-Means} & \lr{augmented} & \lr{RAW} & $5$ & $\max$ \lr{silhouette} & $0.1460$ & $0.5483$ & \lr{---} \\
			\lr{K-Means} & \lr{augmented} & \lr{OPT} & $4$ & $\max$ \lr{silhouette} & $0.1583$ & $0.4627$ & \lr{---} \\
			\midrule
			\lr{DBSCAN} & \lr{no\_augmentation} & \lr{RAW} & $4$ & نزدیک به $4$ + نویز کم & $0.1755$ & $0.2968$ & $0.0174$ \\
			\lr{DBSCAN} & \lr{no\_augmentation} & \lr{OPT} & $4$ & نزدیک به $4$ + نویز کم & $0.2138$ & $0.4263$ & $0.1406$ \\
			\lr{DBSCAN} & \lr{augmented} & \lr{RAW} & $4$ & نزدیک به $4$ + نویز کنترل‌شده & $0.1617$ & $0.4060$ & $0.2338$ \\
			\lr{DBSCAN} & \lr{augmented} & \lr{OPT} & $4$ & نزدیک به $4$ + نویز کنترل‌شده & $0.1571$ & $0.4052$ & $0.2338$ \\
			\midrule
			\lr{OPTICS} & \lr{no\_augmentation} & \lr{RAW} & $4$ & نزدیک به $4$ + \lr{silhouette} & $0.4298$ & $1.0000$ & $0.8142$ \\
			\lr{OPTICS} & \lr{no\_augmentation} & \lr{OPT} & $4$ & نزدیک به $4$ + \lr{silhouette} & $0.5578$ & $0.9929$ & $0.7569$ \\
			\lr{OPTICS} & \lr{augmented} & \lr{RAW} & $2$ & بهترین در شبکه‌ی \lr{Xi} & $0.4869$ & $1.0000$ & $0.9248$ \\
			\lr{OPTICS} & \lr{augmented} & \lr{OPT} & $2$ & بهترین در شبکه‌ی \lr{Xi} & $0.4704$ & $0.4125$ & $0.8843$ \\
			\bottomrule
		\end{tabular}%
	}
\end{table}

\subsubsection{نتیجه‌گیری نهایی}
\begin{itemize}
\item اگر هدف اصلی «بازگرداندن تعداد خوشه‌ی $4$» باشد، در این خروجی‌ها \lr{K-Means} (به‌ویژه در \lr{no\_augmentation} و نیز \lr{augmented-OPT}) و همچنین \lr{DBSCAN} (در همه‌ی حالت‌ها) تعداد خوشه‌ی $4$ را به دست داده‌اند.
\item \lr{OPTICS} در \lr{no\_augmentation} تعداد خوشه‌ی $4$ را می‌دهد، اما با هزینه‌ی نویز بسیار زیاد. بنابراین باید در تفسیر، همزمان به $\text{NoiseRatio}$ توجه کرد (نویز زیاد می‌تواند باعث «خالص شدن» خوشه‌های باقی‌مانده و افزایش ظاهری \lr{purity} روی نقاط غیرنویز شود).
\item در حالت \lr{augmented}، \lr{OPTICS} (با استخراج \lr{Xi}) به $2$ خوشه رسیده است؛ که نشان می‌دهد ساختار چگالیِ داده‌ی افزایش‌یافته برای استخراج \lr{Xi} به شکل فعلی مناسب نبوده یا نیازمند شبکه‌ی پارامتری گسترده‌تر است.
\end{itemize}


