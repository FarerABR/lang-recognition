% !TEX TS-program = xelatex
% !TEX root = ../root/template.tex
% Document class and spacing
\renewcommand{\baselinestretch}{1.3}

% Packages
%\usepackage[colorlinks=true, linkcolor=blue, citecolor=blue, urlcolor=blue]{hyperref}
\usepackage{graphicx}
\usepackage{amsmath, amssymb,amsthm,listings}
\usepackage{geometry}
\usepackage{setspace}
\usepackage{fancyhdr}
\usepackage{titlesec}
%\usepackage{hyperref}
\usepackage[colorlinks=true, linkcolor=blue, citecolor=blue, urlcolor=blue]{hyperref}
\usepackage{xcolor}
\usepackage{titlesec}
\usepackage{tcolorbox}
\usepackage{xepersian}
\usepackage{calc}
\usepackage{eso-pic}
\usepackage{enumitem}
\usepackage{subcaption}
\usepackage{booktabs}
\usepackage{xparse}
\usepackage{float}

\usepackage{algorithm}
\usepackage{algpseudocode}

\usepackage{placeins}

\usepackage{tikz}
\usetikzlibrary{arrows.meta, positioning}



% Fonts
\settextfont[BoldFont = *Bd,
  ItalicFont = *It,
  Scale=1,
  BoldItalicFont = *BdIt,
  Path={../font/},
  Extension = .ttf,
  Color=black
]{XB Niloofar}
\setlatintextfont[Scale=1, Color=black]{Times New Roman}

% Colors
\definecolor{darkblue}{rgb}{.204,.353,.541}
\definecolor{lightblue}{RGB}{210, 227, 245}
\definecolor{lightGreen}{RGB}{210, 245, 227}
\definecolor{darkGreen}{rgb}{.190,.280,.200}

% Custom section fonts
\newfontfamily\subsubsectionfont[
  Scale=1,
  Path={../font/},
  Extension = .ttf,
  Color=black
]{XB Titre}

% Section box command
\newcommand{\sectionbox}[1]{%
  \begin{tcolorbox}[colback=lightblue, colframe=lightblue, width=\linewidth, sharp corners]
    \color{darkblue} \eighteenpt #1
  \end{tcolorbox}
}

% Section and subsection formatting
\titleformat{\section}
{\eighteenpt}{} {0em} {\subsubsectionfont \sectionbox}
\titlespacing{\section}{0pt}{0pt}{0pt}
\titleformat{\subsection}
{\color{darkblue}\sixteenpt\bfseries}
{\thesubsection}{1em}
{}

% Page frame margins
\newlength{\PageFrameTopMargin}
\newlength{\PageFrameBottomMargin}
\newlength{\PageFrameLeftMargin}
\newlength{\PageFrameRightMargin}

\setlength{\PageFrameTopMargin}{0.75cm}
\setlength{\PageFrameBottomMargin}{0.75cm}
\setlength{\PageFrameLeftMargin}{0.75cm}
\setlength{\PageFrameRightMargin}{0.75cm}

%\expandafter\def\expandafter\normalsize\expandafter{%
%	\normalsize%
%	\setlength\abovedisplayskip{0pt}%
%	\setlength\belowdisplayskip{5pt}%
%	\setlength\abovedisplayshortskip{0pt}%
%	\setlength\belowdisplayshortskip{0pt}%
%}

\makeatletter

% Page frame drawing
\newlength{\Page@FrameHeight}
\newlength{\Page@FrameWidth}
\AddToShipoutPicture{
  \setlength{\Page@FrameHeight}{\paperheight-\PageFrameTopMargin-\PageFrameBottomMargin}
  \setlength{\Page@FrameWidth}{\paperwidth-\PageFrameLeftMargin-\PageFrameRightMargin}
  \put(\strip@pt\PageFrameLeftMargin,\strip@pt\PageFrameTopMargin){
    {\framebox(\strip@pt\Page@FrameWidth, \strip@pt\Page@FrameHeight){}}
  }
}
\makeatother

% Font size commands
\newcommand{\sixteenpt}{\fontsize{16pt}{18pt}\selectfont}
\newcommand{\eighteenpt}{\fontsize{18pt}{20pt}\selectfont}
\newcommand{\twentytwopt}{\fontsize{22pt}{24pt}\selectfont}
\newcommand{\twentyeightpt}{\fontsize{28pt}{30pt}\selectfont}

\makeatletter
\def\maketitle{
  \begin{center}
    % Row with logos
    \begin{tabular}{ p{3cm} p{7cm} p{3.5cm} }
      \includegraphics[width=3cm]{../img/logo.png} &
      \vspace{-3cm}
      \begin{center}
        {\sixteenpt{\textbf{دانشگاه تهران}}\\
          \sixteenpt{\textbf{پردیس دانشکده‌های فنی}}\\
        \sixteenpt{\textbf{دانشکده مهندسی برق و کامپیوتر}}}
      \end{center}
      &% Empty middle space
      \includegraphics[width=3.5cm]{../img/eng-logo.png} % Right logo
    \end{tabular}

    \vspace{3cm}
    {\twentyeightpt{\textbf{
      یادگیری ماشین
      }}}\\

    \vspace{1.5cm}
    {\twentytwopt{\textbf{پروژه اصلی فاز شماره \assignNum}}}\\

    \vspace{2cm}
    {\sixteenpt{
      نام و نام خانوادگی
    }}\\
    {\sixteenpt{\@author}}\\[2em]

    {\sixteenpt{
      شماره دانشجویی
    }}\\
    {\sixteenpt{\SNum}}\\

    \vspace{3cm}
    {\normalsize\@date}
  \end{center}
}
\makeatother

% emami make equation number
% تعریف شماره‌گذاری سفارشی
\makeatletter
\renewcommand{\theequation}{
%	\text{معادله}
	\arabic{section}
	\ifnum\value{subsection}>0
	.\arabic{subsection}
	\fi
	.\arabic{equation}
}
\makeatother

% شماره‌گذاری فرمول‌ها بر اساس بخش
\numberwithin{equation}{subsection}

\newtheorem{theorem}{قضیه}
\newtheorem*{theorem*}{قضیه}
\newtheorem{lemma}{لم}
\newtheorem{definition}{تعریف}
\newtheorem{proposition}{گزاره}
\newtheorem{remark}{نکته}
\newtheorem{corollary}{نتیجه}

